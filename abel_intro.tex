\documentclass[a4paper]{article}
\usepackage[14pt]{extsizes}
\usepackage[utf8]{inputenc}
\usepackage[russian]{babel}
\usepackage{indentfirst}
\usepackage{misccorr}
\usepackage{graphicx}
\usepackage{amsmath}
\usepackage{setspace}
\usepackage{fancyhdr}
\usepackage{amssymb} %для готических буков
%
\usepackage[left=1.5cm,right=2cm,top=1.5cm,bottom=2.5cm,bindingoffset=0cm, nohead, footskip=10mm]{geometry}
%
\renewcommand{\rmdefault}{ftm}
%\linespread{1.3} %для интервалов
%\setlength{parskip}{1ex plus 0.5ex minus 0.2ex}

\usepackage{color,colortbl}
\definecolor{honey}{RGB}{250, 200, 50}
\begin{document} % начало документа

% НАЧАЛО ТИТУЛЬНОГО ЛИСТА
\begin{center}
\hfill \break
\large{МИНОБРНАУКИ РОССИИ}\\
\footnotesize{ФЕДЕРАЛЬНОЕ ГОСУДАРСТВЕННОЕ БЮДЖЕТНОЕ ОБРАЗОВАТЕЛЬНОЕ УЧРЕЖДЕНИЕ}\\
\footnotesize{ВЫСШЕГО ПРОФЕССИОНАЛЬНОГО ОБРАЗОВАНИЯ}\\
\small{\textbf{«ВОРОНЕЖСКИЙ ГОСУДАРСТВЕННЫЙ УНИВЕРСИТЕТ»}}\\
\hfill \break
Факультет компьютерных наук\\
 \hfill \break
\normalsize{Кафедра цифровых технологий}\\
\hfill\break
\hfill \break
\hfill \break
\hfill \break
\large{Интегрирование семимерной алгебры Ли}\\
\hfill \break
\hfill \break
\hfill \break
\hfill \break
\normalsize{Б1.О.08 Математические методы в современных\\\hfill информационных технологиях
\hfill \break
09.04.02 Информационные системы и технологии}\\
\hfill \break
\hfill \break
\hfill \break
\hfill \break
\hfill \break
\end{center}

\normalsize{
\begin{tabular}{ccl}
Обучающийся & \underline{\hspace{3cm}} &И.С. Господарикова, 1 курс маг., д/о\\
Обучающийся & \underline{\hspace{3cm}} &Л.А. Прохорченко, 1 курс маг., д/о\\
Руководитель & \underline{\hspace{3cm}}&А.В. Лобода, д.ф-м.н., профессор \\\\
\end{tabular}
}\\
\hfill \break
\hfill \break
\hfill \break
\hfill \break
\hfill \break
\hfill \break
\hfill \break
\begin{center} Воронеж 2019 \end{center}
\thispagestyle{empty} % выключаем отображение номера для этой страницы

% КОНЕЦ ТИТУЛЬНОГО ЛИСТА

\newpage

    \tableofcontents % Вывод содержания
\newpage

\section{Формулировка задачи}

Как следует из названия, в данной работе нам нужно рассмотреть некоторый объект, называемый семимерной алгеброй Ли, и найти его интегральную поверхность.

Итак, по заданию мы имеем некоторую алгебру Ли, определённую в четырёхмерном комплексном пространстве и составленную путём сложения двух алгебр Ли: пятимерной и двумерной.

\begin{equation}
\mathfrak{g}_{7} = \mathfrak{g}_{5} + \mathfrak{g}_{2}
\end{equation}

Двумерная алгебра Ли имеет только один возможный вид, что принимается нами как аксиома: это векторное пространство из двух векторных полей, $e_{1}$ и $e_{2}$, коммутатор между которыми равен одному из этих полей, например, $e_{1}$.

Структура пятимерной алгебры Ли несколько сложнее. В нашем случае, вариантом задания определяется закономерность коммутаторов в пятимерной алгебре. Собственно, строка решаемого нами варианта выглядит следующим образом.

\begin{table}[h]
\begin{center}
\begin{tabular}{|c|c|c|c|c|c|c|c|c|c|}
\hline
$[e_{1},e_{2}]$ & $[e_{1},e_{3}]$ & $[e_{1},e_{4}]$ & \cellcolor{honey} $[e_{1},e_{5}]$ & \cellcolor{honey} $[e_{2},e_{3}]$ & $[e_{2},e_{4}]$ & \cellcolor{honey} $[e_{2},e_{5}]$ & $[e_{3},e_{4}]$ & \cellcolor{honey} $[e_{3},e_{5}]$ & \cellcolor{honey} $[e_{4},e_{5}]$ \\
\hline
 $0$ & $0$ & $0$ & $2e_{1}$ &  $e_{1}$ & $0$ & $e_{2}+e_{3}$ & $0$ & $e_{3}$ &  $\beta e_{4}$ \\
\hline
\end{tabular}
\end{center}
\end{table}

При сложении этих двух алгебр вводится сквозная нумерация. Так векторные поля двумерной алегебры, ранее обозначенные как $e_{1}$ и $e_{2}$ в исследуемой семимерной алгебре Ли станут нумероваться как $e_{6}$ и $e_{7}$.

В задании также предполагается, что коммутаторы между парой векторов, один из которых принадлежит пятимерной подалгебре, а другой - двумерной, будут равны нулю. Это можно понять, исходя из соображений того, что общее (семимерное) векторное пространство распадается на два самостоятельных объекта, каждый из которые может быть рассмотрен в качестве алгебры Ли. В таком случае, векторные поля из этих двух самостоятельных объектов просто не могут иметь каких-то взаимосвязей и, как следствие, не коммутируют.

Мы рассматриваем векторные поля в четырёхмерном комплексном пространстве, и каждое из них будет описываться такой формулой:
\begin{equation}
e_{i} = a_{i} \cdot \frac{\partial}{\partial z_{1}} + b_{i} \cdot \frac{\partial}{\partial z_{2}} + c_{i} \cdot \frac{\partial}{\partial z_{3}} + d_{i} \cdot \frac{\partial}{\partial z_{4}}
\end{equation}

Данную запись мы можем сократить, записав только компоненты векторного поля: $e_{i} = (a_{i}(z), b_{i}(z), c_{i}(z), d_{i}(z))$. Здесь $z$ - четырёхмерный комплексный вектор, описываемый набором комплексных переменных  $(z_{1}, z_{2}, z_{3}, z_{4})$.

\newpage
\section{Проверка свойств алгебры Ли}
Как известно, алгебра Ли - это линейное пространство, в котором, дополнительно к операциям сложения векторов и умножения вектора на скаляр, свойственным линейному пространству, определена операция коммутации \cite[с.~35]{Kostrikin}. Для этой операции будут характерны три свойства: билинейность, антикоммутативность, а также свойство удовлетворения тождеству Якоби\cite[с.~38]{Kostrikin}. Перед началом работы с исследуемым объектом, убедимся, что заданная для него операция коммутации соответствует необходимым свойствам, а сам объект действительно является алгеброй Ли. Примем на веру, что сумма двух алгебр ли даёт в результате тоже алгебру Ли. Двумерная часть нашей семимерной алгебры является алгеброй Ли, что утверждается нам с самого начала. Что же касается пятимерного векторного прострнства, для него мы проверим данные свойства.

\subsection{Билинейность и антикоммутативность}
Свойство билинейности заключается в том, что для любых $e_{i}, e_{j} \in \mathfrak{g}_{5}$ выполняется условие $[e_{i}, e_{j}] \in \mathfrak{g}_{5}$. Иными словами, результат коммутатора любых двух векторов из $\mathfrak{g}_{5}$ должен иметь выражение в виде суперпозиции базисных векторных полей $\mathfrak{g}_{5}$. Посмотрев на формулировку задачи, несложно заметить, что любой коммутатор в исследуемом пространстве действительно удовлетворяет поставленному условию.

Что же касается антикоммутативности, это свойство выражается в том, что для любых $e_{i}, e_{j} \in \mathfrak{g}_{5}$ выполняется условие $[e_{i}, e_{j}] = - [e_{j}, e_{i}]$. В варианте задания все закономерности задаются для коммутаторов лишь в одном порядке, поэтому надо полагать, что свойство антикоммутативности принимается как данность.

\subsection{Тождество Якоби}
Свойство удовлетворения тождеству Якоби несколько сложнее и не столь очевидно, поэтому его мы рассмотрим отдельно. Оно выражается следующей формулой:

\begin{equation*}
[[e_{i}, e_{j}], e_{k}] + [[e_{j}, e_{k}], e_{i}] + [[e_{k}, e_{i}], e_{j}] = 0,
\end{equation*}
 - где $e_{i}, e_{j}, e_{k} \in \mathfrak{g}_{5}$. Вообще говоря, предполагается, что каждый вариант выбора тройки $i,j,k$ имеет какие-то свои особенности и должен быть рассмотрен, включая любые варианты выбора, когда в тройку входят одни и те же числа. Однако несложно заметить, что для троек $j,k,i$, $k,i,j$ результат этого выраженния не изменится. Более того, для перестановок, использующих обратный порядок ($k,j,i$,  $j,i,k$ , $i,k,j$), результат будет отличаться только знаком, что не критично для нас, так как мы доказываем равенство данного выражения нулю. Таким образом, мы имеем право рассматривать только лишь одну еднственную перестановку для каждых трёх векторных полей, что значительно облегчает нам жизнь.

С таким серьёзным упрощением, мы можем привести результаты проверки тождества для каждой комбинации.

\begin{equation*}
1,2,3: [0, e_{3}] + [e_{1}, e_{1}] + [0, e_{2}] = 0
\end{equation*}
\begin{equation*}
1,2,4: [0, e_{4}] + [0, e_{1}] + [0, e_{2}] = 0
\end{equation*}
\begin{equation*}
1,2,5: [0, e_{5}] + [e_{2}+e_{3}, e_{1}] + [-2e_{1}, e_{2}] = 0
\end{equation*}
\begin{equation*}
1,3,4: [0, e_{4}] + [0, e_{1}] + [0, e_{3}] = 0
\end{equation*}
\begin{equation*}
1,3,5: [0, e_{5}] + [e_{3}, e_{1}] + [-2e_{1}, e_{3}] = 0
\end{equation*}
\begin{equation*}
2,3,4: [e_{1}, e_{4}] + [0, e_{2}] + [0, e_{3}] = 0
\end{equation*}
\begin{equation*}
2,3,5: [e_{1}, e_{5}] + [e_{3}, e_{2}] + [-e_{2}-e_{3}, e_{3}] = 0
\end{equation*}
\begin{equation*}
2,4,5: [0, e_{5}] + [\beta e_{4}, e_{2}] + [-e_{2}-e_{3}, e_{4}] = 0
\end{equation*}
\begin{equation*}
3,4,5: [0, e_{5}] + [\beta e_{4}, e_{3}] + [-e_{3}, e_{4}] = 0
\end{equation*}

После этих записей, остаётся неочевидной только верность равенства для тройки $2,3,5$, поэтому распишем это равенство ещё немного:

\begin{equation*}
2,3,5: \underbrace{[e_{1}, e_{5}]}_{2e_{1}} + \underbrace{[e_{3}, e_{2}]}_{-e_{1}} + \underbrace{[-e_{2}-e_{3}, e_{3}]}_{-e_{1}-0} = 0
\end{equation*}

Таким образом, тождество Якоби можно считать доказанным для исследуемого векторного пространства, а значит то, что мы называем $\mathfrak{g}_{5}$, действительно является алгеброй Ли. Это, в свою очередь, означает, что исследуемый нами объект $\mathfrak{g}_{7}$ также является алгеброй Ли, и теперь мы можем приступать к непосредственной работе с ним.

\newpage
\section{Поиск векторных полей в явном виде}
Проинтегрировать алгебру Ли означает найти её интегральную гиперповерхность. Нам известно, что искомая поверхность будет удовлетворять системе дифференциальных уравнений в частных производных:

\begin{equation}
  Re\big( e_{k}(\Phi) |_{M}\equiv 0 \big) (k=1,...7).
\end{equation}

Для дальнейшего поиска решений, нам необходимо выразить векторные поля $e_{k}$ в явном виде. Для этого воспользуемся таким знанием: "Если семимерная алгебра векторных полей в четырёхмерном комплексном пространстве имеет невырожденные орбиты и 4-мерную абелеву подалгебру $\mathfrak{h}$, то базис такой подалгебры можно считать имеющим один из видов:

\begin{table}[h]
\begin{center}
\begin{tabular}{c|c|c}
 $e_{1}:(1,0,0,0)$ & $e_{1}:(1,0,0,0)$ & $e_{1}:(1,0,0,0)$ \\
 $e_{2}:(0,1,0,0)$ & $e_{2}:(0,1,0,0)$ & $e_{2}:(0,1,0,0)$ \\
 $e_{3}:(0,0,1,0)$ & $e_{3}:(0,0,1,0)$ & $e_{3}:(a_{3}(z_{4}),b_{3}(z_{4}),0,0)$ \\
 $e_{4}:(0,0,0,1)$ & $e_{4}:(a_{4}(z_{4}),b_{4}(z_{4}),c_{4}(z_{4}),0)$ & $e_{4}:(0,0,1,0)$." \\
\end{tabular}
\end{center}
\end{table}

Собственно, данное утверждение и определяет сценарий нашего дальнейшего поведения: мы будем искать в нашей алгебре абелеву подалгебру, и, затем, пытаться "примерить" на неё различные варианты базисов. В случае, если ни один из них нам не подойдёт, мы сможем со всей уверенностью положить, что данная алгебра Ли не имеет невырожденных орбит, что будет означать для нас невозможность существования искомой интегральной поверхности.

\subsection{Выбор абелевой подалгебры}
Как следует из определения абелевой подалгебры, это такое подмножество векторных полей $\mathfrak{h}$ в $\mathfrak{g}_{7}$, что для любых $e_{i}, e_{j} \in \mathfrak{h}$ будет выполняться $[e_{i}, e_{j}]=0$. В нашем случае будут иметь место только лишь четырёхмерные варианты такой абелевой подалгебры. Туда могут войти вектора $e_{1}$, $e_{4}$, а также на выбор либо $e_{2}$, либо $e_{3}$, но не оба вместе, поскольку между собой они имеют ненулевой коммутатор. Также к ним можно добавить один из векторов $e_{6}$ и $e_{7}$, но, из аналогичных соображений, не оба сразу. В конце цонцов мы остановились на абелевой подалгебре $\mathfrak{h}=\{e_{1}, e_{3}, e_{4}, e_{6}\}$. Такой её выбор не хуже любого другого, и вполне имеет право на существование.

\subsection{Попытка выпрямления первым вариантом базиса}
Итак, начнём с первого, самого простого варианта базиса. Он предполагает для нас присваивание векторам абелевой подалгебры определённого вида, указанного ранее. После этого присваивания, нам нужно будет посредством расчёта коммутаторов найти явный вид векторных полей $e_{2}$, $e_{5}$ и $e_{7}$, не вошедших в подалгебру, а заодно убедиться, не противоречит ли такой явный вид векторных полей заданной условием задачи операции коммутации.

Начнём с векторного поля $e_{2}$. $[e_{1}, e_{2}]$, как следует из таблицы, равен нулю. Опуская некоторые несложные, но довольно громоздкие записи, можем сказать, что коммутатор такого вида, с векторным полем, компонентами которого являются три нуля и одна единица, покажет нам зависимость другого векторного поля в коммутаторе от комплексной переменной, одной из составляющих вектора, на месте которой стоит единица. Применив это знание, можем сказать, что компоненты векторного поля $e_{2}$ все не зависят от переменной $z_{1}$. По аналогичным соображениям, не будут они зависеть и от $z_{3}$, $z_{4}$ (из коммутаторов $[e_{4}, e_{2}]$ и $[e_{6}, e_{2}]$, из которых оба равны нулю). Что же касается коммутатора $[e_{3}, e_{2}] = -e_{1}$, он при полной записи даёт нам такое уравнение:

\begin{equation*}
[e_{3}, e_{2}](f) = \frac{\partial a_{2}(z)}{\partial z_{2}} \cdot \frac{\partial f}{\partial z_{1}} = -\frac{\partial f}{\partial z_{1}}
\end{equation*}

Это означает, что $a_{2}(z)$ примет в данном случае явный вид, который, за опущением некоторых излишних констант, можно описать следующим выражением:
\begin{equation}
e_{2}:(-z_{2} ; B_{2} ; C_{2} ; D_{2});
\end{equation}

- где заглавными буквами обозначены константы. Аналогичным образом получим, что $e_{5}$ и $e_{7}$ в данном случае примут явный вид такого характера:

\begin{equation}
e_{5}:(2\cdot z_{1} ; z_{2} ; \beta \cdot z_{3} ; D_{5});
\end{equation}
\begin{equation}
e_{7}:(A_{7} ; B_{7} ; C_{7} ; z_{4});
\end{equation}

Однако, чтобы удостовериться, что данное решение нам действительно подходит, мы должны проверить ещё и оставшиеся коммутаторы: $[e_{2}, e_{5}]$, $[e_{2}, e_{7}]$ и $[e_{5}, e_{7}]$. Если какой-то из них не будет давать в результате то, что было задано в начальной таблице, то такой явный вид векторных полей нас не устроит. Собственно, у нас и возникает противоречие, в коммутаторе  $[e_{2}, e_{5}]$:

\begin{equation}
  \begin{aligned}
        $[$e_{2}, e_{5}$]$(f) = -2z_{1} \cdot \frac{\partial f}{\partial z_{1}} + B_{2} \cdot \frac{\partial f}{\partial z_{2}} + C_{2} \cdot \frac{\partial f}{\partial z_{3}}
        - z_{2} \cdot (-1) \cdot \frac{\partial f}{\partial z_{1}}
        = \\ = -z_{2} \cdot \frac{\partial f}{\partial z_{1}} + (B_{2}+1) \cdot \frac{\partial f}{\partial z_{2}} + C_{2} \cdot \frac{\partial f}{\partial z_{3}} + D_{2} \cdot \frac{\partial f}{\partial z_{4}}
        = (e_{2}+e_{3})(f).
  \end{aligned}
\end{equation}

 Из этой записи следует противоречие $B_{2}=B_{2}+1$. Таким образом, мы можем сказать, что найденный явный вид нам не подходит, а значит, имеет смысл отбросить такой вариант базиса и рассмотреть другие.

\subsection{Попытка выпрямления вторым вариантом базиса}
В данном варианте базиса первые три вектора $\in \mathfrak{h}$, остаются такими же, а четвёртый принимает вид $e_{6}:(a_{6}(z_{4}), b_{6}(z_{4}),c_{6}(z_{4}), 0)$. Очевидно, что раз у нас не изменились виды векторых полей и результаты их коммутаторов, то расчёты для них покажут то же самое, а именно, что вид искомых векторов будет:
\begin{equation}
e_{2}:(-z_{2} + a_{2}(z_{4}) ; b_{2}(z_{4}) ; c_{2}(z_{4}) ; d_{2}(z_{4}));
\end{equation}

\begin{equation}
e_{5}:(2\cdot z_{1} + a_{5}(z_{4})  ; z_{2} + b_{5}(z_{4})  ; \beta \cdot z_{3} + c_{5}(z_{4})  ; d_{5}(z_{4}));
\end{equation}

\begin{equation}
e_{7}:( a_{7}(z_{4}) ; b_{7}(z_{4}) ; c_{7}(z_{4}) ; d_{7}(z_{4}));
\end{equation}

Далее следует заметить ещё одну вещь: в случае, если мы встречаем явный вид алгебры, в котором хотя бы для шести векторных полей отсутствует какая-либо зависимость от одной из компонент, то есть на одном из мест $a$, $b$, $c$ или $d$ стоит $0$, тогда такой явный вид описывает алгебру с интегральной поверхностью, не зависящей от одной из координат, а это считается вырождением, и нас в качестве решения не устраивает.

Исходя из этого, мы можем сказать, что для нахождения подходящих нам решений требуется, чтобы из $d_{2}$, $d_{5}$ и $d_{7}$ минимум два значения были бы не нулевыми. На этом основании удобно рассмотреть два подслучая: положив одно из этих значений равным нулю и, для другого подслучая, не равным нулю. Допустим, это будет $d_{2}$. Тогда в первом подслучае $d_{5}$ и $d_{7}$ обязаны не раняться нулю. Тогда мы можем "частично выпрямить" одно из полей $e_{5}$ и $e_{7}$ относительно компоненты $d$. Пускай это будет поле $e_{7}$, и примет оно в таком случае вид $e_{7}:(0,0,0,1)$. Коммутаторы с этим вектором становятся довольно-таки простыми. Из коммутаторов $[e_{2}, e_{7}]$ и  $[e_{5}, e_{7}]$
 будет следовать отсутствие зависимости компонент векторных полей $e_{2}$ и  $e_{5}$ от комплексной переменной $z_{4}$. Тогда они могут быть записаны как $e_{2}:(-z_{2} ; B_{2} ; C_{2} ; 0)$ и $
 e_{5}:(2\cdot z_{1} ; z_{2} ; \beta \cdot z_{3} ; D_{5})$. Рассчитав коммутатор между ними, получим:
 \begin{equation}
   \begin{aligned}
         $[$e_{2}, e_{5}$]$(f) = -2z_{1} \cdot \frac{\partial f}{\partial z_{1}} + B_{2} \cdot \frac{\partial f}{\partial z_{2}} + \beta \cdot C_{2} \cdot \frac{\partial f}{\partial z_{3}}
         + z_{2} \cdot \frac{\partial f}{\partial z_{1}}
         = \\ = -z_{2} \cdot \frac{\partial f}{\partial z_{1}} + (B_{2}+1) \cdot \frac{\partial f}{\partial z_{2}} + C_{2} \cdot \frac{\partial f}{\partial z_{3}}.
   \end{aligned}
 \end{equation}

Отсюда снова следует противоречие $B_{2}=B_{2}+1$, из которого мы делаем вывод, что данный подслучай не даёт нам удовлтворительных решений.

Рассмотрим другой подслучай. Если $d_{2} \neq 0$, то мы можем применить к нему процедуру частичного выпрямления относительно  компоненты $d$. Тогда векторное поле $e_{2}$ примет вид $(-z_{2}, 0,0,1)$. Рассчитаем коммутаторы $[e_{2}, e_{7}]$ и  $[e_{2}, e_{6}]$. Это укажет нам, что $e_{6}$ и $e_{7}$ в данном случае обязаны будут иметь виды $(-B_{6}\cdot z_{4}; B_{6}; C_{6}; 0)$ и $(-B_{7}\cdot z_{4}; B_{7}; C_{7}; D_{7})$ соответственно.
Распишем также комутатор $[e_{6}, e_{7}]$, для которого получается:

\begin{equation}
        [e_{6}, e_{7}](f) = D_{7} \cdot B_{6} \cdot \frac{\partial f}{\partial z_{1}} = - B_{6} \cdot z_{4} \cdot \frac{\partial f}{\partial z_{1}} + B_{6} \cdot \frac{\partial f}{\partial z_{2}} + C_{6} \cdot \frac{\partial f}{\partial z_{3}}.
\end{equation}

Отсюда видно, что для выполнения такого равенства $B_{6}$ и $С_{6}$ пришлось бы положить равными нулю, что привело бы к вырождению векторного поля $e_{6}$. Значит, танный подслучай также не даёт нам решений, а вместе с ним и весь вариант базиса нам не подходит. Рассмотрим оставшийся третий вариант.

\subsection{Попытка выпрямления третьим вариантом базиса}
В данном варианте базиса первые два вектора $\in \mathfrak{h}$, остаются неизменными, для третьего можно сказать, что $e_{4}:(a_{4}(z_{4}), b_{4}(z_{4}), 0, 0)$, а четвёртый принимает бывший вид третьего, то есть $e_{6}:(0,0,1,0)$. Зависимости от $z_{1}$ и $z_{2}$ не изменятся. От переменной $z_{3}$, представленной теперь в векторном поле $e_{6}$, компоненты полей $e_{2}$ и $e_{5}$ зависеть не будут, а у $e_{7}$ определится зависимость в третьей компоненте.

В результате таких рассуждений для искомых векторов будет получен следующий вид:
\begin{equation}
e_{2}:(-z_{2} + a_{2}(z_{4}) ; b_{2}(z_{4}) ; c_{2}(z_{4}) ; d_{2}(z_{4}));
\end{equation}

\begin{equation}
e_{5}:(2\cdot z_{1} + a_{5}(z_{4}) ; z_{2} + b_{5}(z_{4}) ; c_{5}(z_{4}) ; d_{5}(z_{4}));
\end{equation}

\begin{equation}
e_{7}:( a_{7}(z_{4}) ; b_{7}(z_{4}) ; z_{3} + c_{7}(z_{4}) ; d_{7}(z_{4}));
\end{equation}

Аналогично с предыдущим вариантом, теперь мы можем рассмотреть два подслучая: $d_{2}=0$ и $d_{2}\neq 0$. В первом подслучае можем частично выпрямить $e_{7}$ до $(0,0,z_{3},1)$. По аналогичному сценарию, там получается противоречие $B_{2}=B_{2}+1$, а следовательно такой подслучай не даёт удобоваримых решений.

Что же касается другого подслучая, тут мы частично выпрямляем $e_{2}$ до $(-z_{2},0,0,1)$. Здесь ситуация получаеся более интересная, и в конце концов мы даже имеем некоторый явный вид векторных полей исследуемой алгебры, котрый, вроде бы, не имеет противоречий. Выглядит он так:


\begin{equation}
  \left \{ \begin{array}{}
  $e_{1}=(1;0;0;0);$ \\
  $e_{3}=(0;1;0;0);$ \\
  $e_{6}=(0;0;1;0);$ \\
  $e_{2}=(-z_{2} ; 0 ;  0 ; 1); $ \\
  $e_{7}=(0 ; 0;  z_{3} ; 0); $ \\
  $e_{5}=(2 \cdot z_{1} -\frac{z_{4}^{2}}{2} + A_{5} \cdot z_{4} + A_{5}' ; z_{2} + z_{4} + B_{5} ; 0 ;  z_{4} + D_{5});$ \\
  $e_{4}=(B_{4} \cdot z_{4} + A_{4}; B_{4} ;  0 ; 0). $ \\
   \end{array}
   \right.
\end{equation}

Теперь, имея явное выражение для каждого из векторных полей, можно наконец-то приступать к главной задаче - интегрированию исследуемой алгебры Ли.

\section{Интегрирование алгебры Ли}
Как уже говорилось ранее, известно, что искомая поверхность будет удовлетворять системе дифференциальных уравнений в частных производных, описываемой выражением (3). Зная, что производная комплексной функции может быть расписана как $\frac{df}{dz} = \frac{1}{2}(\frac{df}{dx}+i \cdot \frac{df}{dy})$, приступим к поиску системы дифференциальных уравнений.

Для первых векторных полей $e_{1}$, $e_{3}$ и $e_{6}$, получим выражения $\frac{\partial \Phi}{\partial x_{1}}=0$, $\frac{\partial \Phi}{\partial x_{2}}=0$ и $\frac{\partial \Phi}{\partial x_{3}}=0$ соответственно. Это значит, что для $i=1, 2, 3$ будет справедливо $\frac{\partial \Phi}{\partial y_{i}} \neq 0$.
Тогда мы можем сказать, что, например, $\frac{\partial \Phi}{\partial y_{1}} = -1$, а $\Phi$ можно выразить в неявном виде как $\Phi = -y_{1} + F$. Исходя из этого, рассчитаем остальные выражения.

\begin{equation*}
    Re\Big(e_{2}(\Phi)\Big) = Re\Big( (-x_{2}-i\cdot y_{2}) \cdot \frac{1}{2} \cdot ( \underbrace{\frac{\partial \Phi}{\partial x_{1}}}_{0} + i \cdot \underbrace{\frac{\partial \Phi}{\partial y_{1}}}_{-1})
     + \frac{1}{2} \cdot ( \frac{\partial F}{\partial x_{4}} + i \cdot \frac{\partial F}{\partial y_{4}}) \Big) = -\frac{1}{2} \cdot y_{2} + \frac{1}{2} \cdot \frac{\partial F}{\partial x_{4}} = 0
\end{equation*}

То есть для $e_{2}$ получим $\Big(\frac{\partial F}{\partial x_{4}}\Big) = y_{2}$.

\begin{equation*}
    Re\Big(e_{7}(\Phi)\Big) = Re\Big( (x_{3}+i\cdot y_{3}) \cdot \frac{1}{2} \cdot ( \underbrace{\frac{\partial \Phi}{\partial x_{3}}}_{0} + i \cdot \frac{\partial \Phi}{\partial y_{3}}) \Big)
     = - \frac{1}{2} \cdot y_{3} \cdot \frac{\partial \Phi}{\partial y_{3}} = 0
\end{equation*}

То есть для $e_{2}$ получим $\Big(\frac{\partial F}{\partial y_{3}}\Big) \cdot y_{3} = 0$.

По сути, если $y_{3}$ является в данном случае некоторой вещественной переменной, полученное выражение означает, функция $\Phi$ от этой переменной не завистит. В сочетании с установленным выше фактом её независимости от $x_{3}$, это значит, что функция полностью не зависит от переменной $z_{3}$, что говорит о вырождении искомой интегральной поверхности. Отсюда можно заключить, что исследуемая нами алгебра Ли не имеет невырожденной интегральной поверхности.

Таким образом, несмотря на то, что мы нашли непротиворечивый явный вид векторных полей исследуемого пространства, на поверку оказалось, что его интегральная поверхность является вырожденной. Заметим также, что другого непротиворечивого явного вида для данной алгебры не найдётся, поскольку мы уже рассмотрели все три предлагаемых варианта базиса.

За сим ответом на задачу интегрирования данной алгебры Ли будет: "Исследуемая алгебра Ли не имеет невырожденных интегральных поверхностей".
\newpage
\addcontentsline{toc}{section}{Список литературы}
\bibliographystyle{utf8gost705u}  %% стилевой файл для оформления по ГОСТу
\bibliography{biblio}
\end{document}  % КОНЕЦ ДОКУМЕНТА !
