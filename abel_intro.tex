\documentclass[a4paper]{article}
\usepackage[14pt]{extsizes} % для того чтобы задать нестандартный 14-ый размер шрифта
\usepackage[utf8]{inputenc}
\usepackage[russian]{babel}
\usepackage{setspace,amsmath}
\usepackage[left=20mm, top=15mm, right=15mm, bottom=15mm, nohead, footskip=10mm]{geometry} % настройки полей документа
\begin{document} % начало документа

% НАЧАЛО ТИТУЛЬНОГО ЛИСТА
\begin{center}
\hfill \break
\large{МИНОБРНАУКИ РОССИИ}\\
\footnotesize{ФЕДЕРАЛЬНОЕ ГОСУДАРСТВЕННОЕ БЮДЖЕТНОЕ ОБРАЗОВАТЕЛЬНОЕ УЧРЕЖДЕНИЕ}\\
\footnotesize{ВЫСШЕГО ПРОФЕССИОНАЛЬНОГО ОБРАЗОВАНИЯ}\\
\small{\textbf{«ВОРОНЕЖСКИЙ ГОСУДАРСТВЕННЫЙ УНИВЕРСИТЕТ»}}\\
\hfill \break
\textit{Факультет компьютерных наук}\\
 \hfill \break
\normalsize{Кафедра цифровых технологий}\\
\hfill\break
\hfill \break
\hfill \break
\hfill \break
\large{Интегрирование семимерной алгебры Ли}\\
\hfill \break
\hfill \break
\hfill \break
\normalsize{Б1.О.08 Математические методы в современных\\\hfill информационных технологиях
\hfill \break
09.04.02 Информационные системы и технологии}\\
\hfill \break
\hfill \break
\hfill \break
\hfill \break
\hfill \break
\hfill \break
\hfill \break
\hfill \break
\hfill \break
\end{center}

\normalsize{
\begin{tabular}{ccl}
Обучающийся & \underline{\hspace{3cm}} &И.С. Господарикова, 1 курс маг., д/о\\
Обучающийся & \underline{\hspace{3cm}} &Л.А. Прохорченко, 1 курс маг., д/о\\
Руководитель & \underline{\hspace{3cm}}&А.В. Лобода, д.ф-м.н., профессор \\\\
\end{tabular}
}\\
\hfill \break
\hfill \break
\hfill \break
\hfill \break
\hfill \break
\hfill \break
\hfill \break
\begin{center} Воронеж 2019 \end{center}
\thispagestyle{empty} % выключаем отображение номера для этой страницы

% КОНЕЦ ТИТУЛЬНОГО ЛИСТА

\newpage

    \tableofcontents % Вывод содержания
\newpage

\newpage
\section{Введение}
Первый абзац о \cite{Vasylenko92} всякой непонятной \cite{Afanasyev92} информации.
\newpage

\subsection{Подвведение}
\newpage

\addcontentsline{toc}{section}{Список литературы}
\bibliographystyle{utf8gost705u}  %% стилевой файл для оформления по ГОСТу
\bibliography{biblio}
\end{document}  % КОНЕЦ ДОКУМЕНТА !
