\documentclass{article}
\usepackage[12pt]{extsizes}
\usepackage[utf8]{inputenc}
\usepackage[russian]{babel}
\usepackage{indentfirst}
\usepackage{misccorr}
\usepackage{graphicx}
\usepackage{amsmath}
\usepackage{setspace}
\usepackage[left=1cm,right=1cm,top=1cm,bottom=2cm,bindingoffset=0cm]{geometry}
\usepackage{fancyhdr}
\usepackage{color,colortbl}

\begin{document}

\newpage
\subsection*{Вариант 3}

При рассмотрении этого варианта абелева подалгебра имеет вид:
\begin{flushleft}
\begin{tabular}{c}
$e_{1}=\{1,0,0,0\};$ \\
$e_{3}=\{0,1,0,0\};$ \\
$e_{4}=\{a_{4}(z_{4}), b_{4}(z_{4}), 0 ,0\};$  \\
$e_{6}=\{0,0,1,0\};$ \\
\end{tabular}
\end{flushleft}

\begin{equation*}
\begin{aligned}
$[$e_{6}, e_{7}$]$(f)
 = \frac{\partial a_{7}(z)}{\partial z_{3}} \cdot \frac{\partial f}{\partial z_{1}}
  + \frac{\partial b_{7}(z)}{\partial z_{3}} \cdot \frac{\partial f}{\partial z_{2}}
   + \frac{\partial c_{7}(z)}{\partial z_{3}} \cdot \frac{\partial f}{\partial z_{3}}
    + \frac{\partial d_{7}(z)}{\partial z_{3}} \cdot \frac{\partial f}{\partial z_{4}}
     = \frac{\partial f}{\partial z_{3}}.
\end{aligned}
\end{equation*}

Видно, что составляющие выкторного поля $e_{7}$ : $a_{7}$, $b_{7}$ и $d_{7}$ - не имеют зависимости от переменной $z_{3}$, а $c_{7}$ будет содержать слагаемое $z_{3}$ в числе прочих. Таким образом, до расчёта коммутаторов, связанных с векторным полем $e_{4}$ для искомых векторных полей мы можем сказать следующее:

\begin{equation*}
\begin{gathered}
\fbox{ e_{2}=\{ -z_{2} + a_{2}(z_{4}); b_{2}(z_{4}) ; c_{2}(z_{4}); d_{2}(z_{4})\} }
\end{gathered}
\end{equation*}

\begin{equation*}
\begin{gathered}
\fbox{ e_{5}=\{ 2 \cdot z_{1} + a_{5}(z_{4}); z_{2} + b_{5}(z_{4}); c_{5}(z_{4});  d_{5}(z_{4})\} }
\end{gathered}
\end{equation*}

\begin{equation*}
\begin{gathered}
\fbox{e_{7}=\{ a_{7}(z_{4}); b_{7}(z_{4});  z_{3} + c_{7}(z_{4}); d_{7}(z_{4})\}}
\end{gathered}
\end{equation*}

Нам известно, что если хотя бы одна координата шести из семи полей равна нулю, то это сведетельствует о вырождении исходной интегральной поверхности. Рассмотрим два подслучая $e_{2} = 0$ и $e_{2} \neq 0$:
% Для первого подслучая ($e_{2} \eq 0$) невырожденное решение достижимо только при $d_{5} \neq 0$ и $d_{7} \neq 0$, а во втором ($d_{2} \neq 0$)

\begin{equation}
\begin{gathered}
$e_{2} = 0 \rightarrow $
\left\{ \begin{array}{}
$d_{5} \neq 0$
\\
$d_{7} \neq 0$
\end{array}\right.
\end{gathered}
\end{equation}

\begin{equation*}
\begin{gathered}
$e_{2} \neq 0 \rightarrow $
\left[ \begin{array}{}
$d_{5} \neq 0$ и $\forall d_{7}$
\\
$d_{7} \neq 0$ и $\forall d_{5}$
\end{array}\right.
\end{gathered}
\end{equation*}

\subsubsection*{Подслучай 1}

Проведём частичное выпрямление $e_{2}$ относительно переменной $z_{4}$. Для этого положим $a_{2}(z_{4}) = 0, b_{2}(z_{4}) = 0, c_{2}(z_{4}) = 0, d_{2}(z_{4}) = 1$:
% \begin{equation*}
% \begin{gathered}
% \end{gathered}
% \end{equation*}
\begin{equation*}
\begin{gathered}
e_{2} \rightarrow e_{2}=\{-z_{2}, 0, 0, 1\}
\end{gathered}
\end{equation*}

Используя это, найдём шесть оставшихся коммутаторов: $[e_{2}, e_{4}]$, $[e_{5}, e_{4}]$, $[e_{7}, e_{4}]$, а также $[e_{2}, e_{7}]$, $[e_{2}, e_{5}]$ и $[e_{7}, e_{5}]$.

\begin{equation*}
\begin{gathered}
$[$e_{2}, e_{4}$]$(f) = \Big(\frac{\partial a_{4}(z_{4})}{\partial z_{4}} \cdot \frac{\partial f}{\partial z_{1}}
+ \frac{\partial b_{4}(z_{4})}{\partial z_{4}} \cdot \frac{\partial f}{\partial z_{2}}\Big)
- b_{4}(z_{4}) \cdot \frac{\partial (-z_{2})}{\partial z_{2}} \cdot \frac{\partial f}{\partial z_{1}}
= \frac{\partial f}{\partial z_{1}} \cdot \Big(\frac{\partial a_{4}(z_{4})}{\partial z_{4}} + b_{4}(z_{4})\Big)
+ \frac{\partial f}{\partial z_{2}} \cdot \frac{\partial b_{4}(z_{4})}{\partial z_{4}} = 0
\end{gathered}
\end{equation*}

\begin{equation*}
% \begin{aligned}
%   \fbox{
%     \left\{ \begin{array}{}
%      % \frac{\partial a_{4}(z_{4})}{\partial z_{4}} = -b_{4}(z_{4})
%      \frac{\partial a_{4}(z_{4})}{\partial z_{4}} = -b_{4}(z_{4})
%      % \\
%      % \frac{\partial b_{4}(z_{4})}{\partial z_{4}} = 0
%      \end{array}\right
%    }
% \end{aligned}
\begin{aligned}
\fbox{
  \left\{ \begin{array}{}
   \Big(d_{2} + D_{2}\Big) \cdot \Big(\frac{\partial a_{4}}{\partial z_{4}}\Big) = b_{4}
   % \Big(\frac{}{}{A}{B}\Big) = -b_{4}(z_{4})
   \\
   \Big(d_{2} + D_{2}\Big) \cdot \Big(\frac{\partial b_{4}}{\partial z_{4}}\Big) = 0
   \end{array}\right.
   }
\end{aligned}
\end{equation*}


\subsubsection*{Подслучай 2}
Окей, теперь это гавно прямо рядом. Ток руку протяни -- уже вляпался!

% \begin{equation*}
% \begin{gathered}
% \fbox{ e_{5}=\{ 2 \cdot z_{1} + a_{5}(z_{4}); z_{2} + b_{5}(z_{4}); c_{5}(z_{4});  d_{5}(z_{4})\} }
% \end{gathered}
% \end{equation*}

% \begin{equation*}
% \begin{aligned}
%
% \end{aligned}
% \end{equation*}

\end{document}
